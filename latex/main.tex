% !TeX program = lualatex
% Für echtes Arial/Calibri: lualatex/xelatex + biber verwenden.
% Fallback mit pdflatex funktioniert (sans: TeX Gyre Heros).

\documentclass[11pt,a4paper]{article}

% ---------- Pakete: Sprache, Schrift, Mikrotypografie ----------
\usepackage[a4paper,margin=2cm]{geometry} % 2,00 cm Ränder
\usepackage[ngerman,english]{babel}       % Silbentrennung DE/EN
\usepackage{microtype}                    % Mikrotypografie (Silbentrennung/Zeilenumbrüche)
\usepackage[T1]{fontenc}
\usepackage[utf8]{inputenc}
\usepackage{iftex}
\ifPDFTeX
% Fallback: pdfLaTeX (Arial-ähnlich)
\usepackage[scale=0.95]{tgheros}   % TeX Gyre Heros ~ Helvetica/Arial
\renewcommand{\familydefault}{\sfdefault}
\else
% Echte Systemschriften mit Lua/XeLaTeX:
\usepackage{fontspec}
\setmainfont{Arial}[
Ligatures=TeX,
Scale=1.0
]
% Optionaler alternativer Sans-Font:
% \setmainfont{Calibri}[Ligatures=TeX,Scale=1.0]
\fi

% ---------- Zeilen & Absätze ----------
\usepackage{setspace}
\onehalfspacing                           % 1,5-zeilig
\setlength{\parindent}{0pt}               % kein Einzug
\setlength{\parskip}{6pt}                 % 6 pt Abstand nach Absatz

% ---------- Mathe, Grafik, Tabellen ----------
\usepackage{amsmath,amssymb}
\usepackage{graphicx}
\usepackage{booktabs}
\usepackage{siunitx}
\sisetup{
	locale = DE,
	detect-all
}                   % Sans-Serif in siunitx
\usepackage{csquotes}                     % saubere Anführungszeichen / Blockzitate
\usepackage{mwe}                          % Beispielbilder (example-image*)

% ---------- Bild-/Tabellenunterschriften (10 pt) ----------
\usepackage{caption}
\captionsetup{font=footnotesize,labelfont=bf,labelsep=colon}
\addto\captionsngerman{\renewcommand{\figurename}{Abb.}}
\addto\captionsngerman{\renewcommand{\tablename}{Tab.}}
\newcommand{\source}[1]{\caption*{\footnotesize #1}} % Quellenangabe 10 pt

% ---------- Überschriften: Größen/Abstände/Nummerntiefe ----------
\usepackage{titlesec}
% H1: 16 pt, 12/12
\titleformat{\section}{\bfseries\fontsize{16pt}{18pt}\selectfont}{\thesection}{0.6em}{}
\titlespacing*{\section}{0pt}{12pt}{12pt}
% H2: 14 pt, 12/6
\titleformat{\subsection}{\bfseries\fontsize{14pt}{16pt}\selectfont}{\thesubsection}{0.6em}{}
\titlespacing*{\subsection}{0pt}{12pt}{6pt}
% H3: 11 pt, 12/6
\titleformat{\subsubsection}{\bfseries\fontsize{11pt}{13pt}\selectfont}{\thesubsubsection}{0.6em}{}
\titlespacing*{\subsubsection}{0pt}{12pt}{6pt}
\setcounter{secnumdepth}{3}
\setcounter{tocdepth}{3}

% Jede Section (Ebene 1) startet auf neuer Seite
\usepackage{etoolbox}
\pretocmd{\section}{\clearpage}{}{}

% ---------- Inhaltsverzeichnis: nur Ebene 1 fett ----------
\usepackage{tocloft}
\renewcommand{\cftsecfont}{\bfseries}
\renewcommand{\cftsecpagefont}{}

% ---------- Fußnoten explizit 10 pt ----------
\makeatletter
\renewcommand\footnotesize{\@setfontsize\footnotesize{10pt}{12pt}}
\makeatother

% ---------- Seitenzahlen: zentriert im Fuß ----------
\usepackage{fancyhdr}
\pagestyle{fancy}
\fancyhf{}
\cfoot{\thepage}

% ---------- Hyperlinks schwarz (keine Farben/Rahmen) ----------
\usepackage[hidelinks]{hyperref}

% ---------- APA7 Literatur (Biber) + deutscher Mapping ----------
\usepackage[
style=apa,
backend=biber,
sorting=nyt,
uniquename=init,
maxcitenames=2, % "et al." ab 3
maxbibnames=99,
doi=true,
url=true,
dateabbrev=false, % Vollständige Datumsangaben
eprint=false, % Unterdrücke eprint-Felder wenn DOI vorhanden
isbn=false, % ISBN normalerweise nicht in APA7
giveninits=true % Nur Initialen für Vornamen
]{biblatex}
\DeclareLanguageMapping{ngerman}{ngerman-apa}

% Zusätzliche APA7-Konfigurationen
\ExecuteBibliographyOptions{maxbibnames=999} % Alle Autoren im Literaturverzeichnis
\ExecuteBibliographyOptions{giveninits=true} % Nur Initialen
\ExecuteBibliographyOptions{uniquename=init} % Eindeutigkeit durch Initialen

% DOI-Formatierung anpassen
\DeclareFieldFormat{doi}{%
  \mkbibacro{DOI}\addcolon\space
  \ifhyperref
    {\href{https://doi.org/#1}{\nolinkurl{#1}}}
    {\nolinkurl{#1}}}

% Hängender Einzug 1.27 cm, 1,5-zeilig wie Text
\setlength{\bibhang}{1.27cm}
\defbibenvironment{bibliography}
{\list
    {\printtext[labelnumberwidth]{\printfield[labelnumberwidth]{labelnumber}}}
    {\setlength{\leftmargin}{\bibhang}
        \setlength{\itemindent}{-\bibhang}
        \setlength{\itemsep}{\baselineskip} % 1.5-Zeilenabstand wie Text
        \setlength{\parsep}{0pt}}
    \renewcommand*{\makelabel}[1]{##1\hss}}
{\endlist}
{\item}

% Bibliografie-Datei (erstelle workbook_statistik.bib bei Bedarf)
% \addbibresource{workbook_statistik.bib}

% ---------- Automatische Verzeichnisse nur bei ≥3 Einträgen ----------
\usepackage{totcount}
\regtotcounter{figure}
\regtotcounter{table}
\newcommand{\addtoTOC}[1]{\addcontentsline{toc}{section}{#1}}
\newcommand{\printlistsconditional}{%
    % Wirksam nach erneutem LaTeX-Lauf (Zähler aus .aux):
    \ifnum\totvalue{figure}>2
    \renewcommand{\listfigurename}{Abbildungsverzeichnis}
    \listoffigures
    \addtoTOC{Abbildungsverzeichnis}
    \clearpage
    \fi
    \ifnum\totvalue{table}>2
    \renewcommand{\listtablename}{Tabellenverzeichnis}
    \listoftables
    \addtoTOC{Tabellenverzeichnis}
    \clearpage
    \fi
}

% ---------- Blockzitat ≥ 40 Wörter (APA) ----------
\newenvironment{blockzitat}{%
    \begin{quote}\setlength{\leftskip}{1.27cm}\itshape\upshape\mdseries\selectfont
    }{\end{quote}}

% ---------- Meta-Felder für Titelseite ----------
\newcommand{\university}{IU Internationale Hochschule}
\newcommand{\studyprogram}{B.\,Sc.\ Data Science}
\newcommand{\thesistype}{Workbook Statistik}
\newcommand{\papertitle}{Statistische Analyse des ALLBUS 2021 Teildatensatzes}
\newcommand{\authorname}{Jonas Weirauch}
\newcommand{\matno}{10237021}
\newcommand{\address}{Im Wiesengrund 19, 55286 Sulzheim}
\newcommand{\advisor}{Oliver Labs}
\newcommand{\submissiondate}{31.12.2025}

% ============================================================
%                         DOKUMENT
% ============================================================
\begin{document}
    \selectlanguage{ngerman}

    % ---------- Titelblatt (zählt als I, ohne Zahl) ----------
    \pagenumbering{Roman}
    \setcounter{page}{1}
    \begin{titlepage}
        \thispagestyle{empty}
        \begin{center}
            \large \university\\[6pt]
            \studyprogram\\[18pt]
            \textbf{\Large \thesistype}\\[24pt]
            {\bfseries\fontsize{18pt}{20pt}\selectfont \papertitle}\\[24pt]
        \end{center}
        \vspace{6mm}
        \begin{tabular}{@{}ll}
            Autor:            & \authorname \\
            Matrikelnummer:   & \matno \\
            Anschrift:        & \address \\
            Betreuung:        & \advisor \\
            Abgabedatum:      & \submissiondate \\
        \end{tabular}
        \vfill
        \begin{center}
            \footnotesize
            % Optional: Hochschullogo oder weitere Angaben
        \end{center}
    \end{titlepage}

    % ---------- Vorspann (Erklärungen, Abstracts etc.) ----------
    % ============================================================
%                       VORSPANN
% ============================================================

% ---------- Erklärung / Sperrvermerk (optional je nach Arbeit) ----------
\section*{Erklärung / Sperrvermerk}
\addtoTOC{Erklärung / Sperrvermerk}

Hier ggf. die Eigenständigkeits- und Sperrvermerkserklärung gemäß Vorgaben der Hochschule.

\clearpage

% ---------- Danksagung (optional) ----------
\section*{Danksagung}
\addtoTOC{Danksagung}

Optionaler Text für Danksagungen.

\clearpage

% ---------- Abstracts (Deutsch & Englisch, je ca. 200 Wörter) ----------
\section*{Abstract (Deutsch)}
\addtoTOC{Abstract (Deutsch)}

Kurzfassung der Arbeit (ca. 200 Wörter): Problemstellung, Methode, Ergebnisse, Implikationen.

\clearpage

\begin{otherlanguage*}{english}
    \section*{Abstract (English)}
    \addtoTOC{Abstract (English)}
    
    Abstract (approx. 200 words): problem, method, results, implications.
\end{otherlanguage*}

\clearpage


    % ---------- Inhaltsverzeichnis ----------
    \renewcommand{\contentsname}{Inhaltsverzeichnis}
    \tableofcontents
    \clearpage

    % ---------- Abbildungs-/Tabellenverzeichnis (nur bei ≥3) ----------
    \printlistsconditional

    % ---------- Abkürzungsverzeichnis ----------
    \section*{Abkürzungsverzeichnis}
    \addtoTOC{Abkürzungsverzeichnis}
    \begin{tabular}{@{}ll}
        \textbf{ALLBUS} & Allgemeine Bevölkerungsumfrage der Sozialwissenschaften \\
        \textbf{ANOVA}  & Analysis of Variance (Varianzanalyse) \\
        \textbf{CI}     & Confidence Interval (Konfidenzintervall) \\
        \textbf{SD}     & Standard Deviation (Standardabweichung) \\
    \end{tabular}
    \clearpage

    % ---------- Hauptteil: arabische Seitenzahlen ab "Einleitung" ----------
    \pagenumbering{arabic}
    \setcounter{page}{1}

    % ============================================================
    %                    INHALTLICHE ABSCHNITTE
    % ============================================================
    
    % ============================================================
%                       EINLEITUNG
% ============================================================

\section{Einleitung}

Dieses Workbook dokumentiert die statistische Analyse des ALLBUS 2021 Teildatensatzes 
im Rahmen des Moduls „Deskriptive und Inferenzstatistik" an der IU Internationale Hochschule 
im Wintersemester 2025.

Der ALLBUS (Allgemeine Bevölkerungsumfrage der Sozialwissenschaften) ist eine repräsentative 
Querschnittserhebung, die seit 1980 Einstellungen, Verhaltensweisen und Sozialstruktur der 
Bevölkerung in Deutschland erhebt. Der vorliegende Teildatensatz aus dem Jahr 2021 umfasst 
29 ausgewählte Variablen.

\subsection{Zielsetzung}

Ziel dieser Arbeit ist die systematische Anwendung deskriptiver und inferenzstatistischer 
Methoden auf reale Umfragedaten. Im Einzelnen werden folgende Analysen durchgeführt:

\begin{enumerate}
    \item \textbf{Univariate Deskription:} Analyse einzelner Variablen nach Skalenniveau
    \item \textbf{Bivariate Deskription:} Untersuchung von Zusammenhängen zwischen zwei Variablen
    \item \textbf{Inferenzstatistik:} Konfidenzintervalle und Hypothesentests
    \item \textbf{Regression:} Modellierung linearer Zusammenhänge
    \item \textbf{ANOVA:} Vergleich von Mittelwerten über mehrere Gruppen
\end{enumerate}

\subsection{Aufbau der Arbeit}

Nach dieser Einleitung folgt in Kapitel~\ref{sec:daten} eine Beschreibung des Datensatzes 
und der verwendeten Variablen. Die Kapitel~\ref{sec:block1} bis \ref{sec:block5} dokumentieren 
die fünf Analyseblöcke gemäß Aufgabenstellung. Kapitel~\ref{sec:diskussion} diskutiert die 
Ergebnisse kritisch, bevor Kapitel~\ref{sec:fazit} die Arbeit zusammenfasst.

Alle statistischen Analysen wurden in Python durchgeführt (Version 3.12.6). Die verwendeten 
Pakete umfassen \texttt{pandas}, \texttt{numpy}, \texttt{matplotlib}, \texttt{scipy} und 
\texttt{statsmodels}.

    % ============================================================
%                    DATENBESCHREIBUNG
% ============================================================

\section{Datenbeschreibung}
\label{sec:daten}

\subsection{Der ALLBUS 2021 Teildatensatz}

Der verwendete Datensatz ist ein Auszug aus dem ALLBUS 2021 (ZA5284, Version 1.1.0) und 
enthält 29 Variablen aus verschiedenen Themenbereichen:

\begin{itemize}
    \item Soziodemografische Merkmale (Alter, Geschlecht, Bildung, Einkommen)
    \item Politische und wirtschaftliche Einschätzungen
    \item Werthaltungen und Einstellungen
    \item Gesundheit und Lebenssituation
\end{itemize}

Die Stichprobe umfasst \texttt{N = [ANZAHL]} Befragte aus der deutschsprachigen 
Wohnbevölkerung ab 18 Jahren.

\subsection{Verwendete Variablen}

Gemäß der individuellen Variablenzuteilung werden in dieser Arbeit folgende Variablen analysiert:

\subsubsection{Block 1: Univariate Deskription}

\begin{itemize}
    \item \textbf{mc04} (nominal): Ausländer im Freundeskreis
    \item \textbf{ep04} (ordinal): Wirtschaftslage in Deutschland in 1 Jahr
    \item \textbf{hhinc} (metrisch): Haushaltsnettoeinkommen in Euro
\end{itemize}

\subsubsection{Block 2: Bivariate Deskription}

\begin{itemize}
    \item \textbf{ep01} (ordinal): Aktuelle Wirtschaftslage in Deutschland
    \item \textbf{fe14} (ordinal): Erziehungsziel „beliebt sein"
    \item \textbf{xt10} (metrisch): Interviewdauer in Minuten
    \item \textbf{age} (metrisch): Alter in Jahren
\end{itemize}

\subsubsection{Block 3: Inferenzstatistik}

\begin{itemize}
    \item \textbf{lm02} (metrisch): Tägliche Fernsehdauer in Minuten
    \item \textbf{sex} (nominal): Geschlecht (für Gruppenvergleich)
\end{itemize}

\subsubsection{Block 4: Regression}

\begin{itemize}
    \item \textbf{age} (metrisch): Alter in Jahren (Prädiktor)
    \item \textbf{hhinc} (metrisch): Haushaltsnettoeinkommen (Zielvariable)
\end{itemize}

\subsubsection{Block 5: ANOVA}

\begin{itemize}
    \item \textbf{hs01} (ordinal): Gesundheitszustand (Gruppierungsfaktor)
    \item \textbf{gd02} (metrisch): Wohndauer im Ort in Jahren
\end{itemize}

\subsection{Datenbereinigung}

Vor den Analysen wurden folgende Bereinigungsschritte durchgeführt:

\begin{enumerate}
    \item Behandlung fehlender Werte (Sondercodes wie -99, -98, -42 etc.)
    \item Konvertierung von Datentypen (numerisch vs. kategorial)
    \item Plausibilitätsprüfungen (Wertebereich, Ausreißer)
    \item Speicherung als bereinigte CSV-Datei (\texttt{allbus\_clean.csv})
\end{enumerate}

Details zur Datenbereinigung finden sich im Python-Skript \texttt{src/data\_prep.py}.

    % ============================================================
%          BLOCK 1: UNIVARIATE DESKRIPTION
% ============================================================

\section{Univariate Deskription und Skalenniveaus}
\label{sec:block1}

\subsection{Aufgabe 1b: Skalenniveaus}

Die Wahl geeigneter statistischer Methoden hängt wesentlich vom Skalenniveau der Variablen ab. 
Man unterscheidet drei grundlegende Skalenniveaus:

\subsubsection{Nominalskalenniveau}

\textbf{Definition:} Variablen mit Nominalskalenniveau besitzen Kategorien ohne natürliche Ordnung. 
Es kann nur Gleichheit oder Verschiedenheit festgestellt werden.

\textbf{Beispiel:} Variable \texttt{mc04} – „Haben Sie Ausländer im Freundeskreis?" mit den 
Kategorien „Ja", „Nein", „Weiß nicht". Diese Kategorien haben keine sinnvolle Reihenfolge.

\textbf{Zulässige Statistiken:}
\begin{itemize}
    \item Häufigkeiten (absolut und relativ)
    \item Modus (häufigste Kategorie)
    \item Balken- oder Kreisdiagramme
\end{itemize}

\subsubsection{Ordinalskalenniveau}

\textbf{Definition:} Variablen mit Ordinalskalenniveau besitzen geordnete Kategorien. 
Die Abstände zwischen den Kategorien sind jedoch nicht interpretierbar.

\textbf{Beispiel:} Variable \texttt{ep04} – „Wie wird sich die wirtschaftliche Lage in Deutschland 
in einem Jahr entwickeln?" mit Kategorien von „wesentlich besser" bis „wesentlich schlechter". 
Die Kategorien sind klar geordnet, aber der Abstand zwischen „besser" und „wesentlich besser" 
ist nicht quantifizierbar.

\textbf{Zulässige Statistiken:}
\begin{itemize}
    \item Alle Statistiken des Nominalniveaus
    \item Median (mittlere Kategorie)
    \item Perzentile
    \item Rangkorrelationen (Spearman, Kendall)
\end{itemize}

\subsubsection{Metrisches Skalenniveau}

\textbf{Definition:} Variablen mit metrischem Skalenniveau (Intervall- oder Verhältnisskala) 
besitzen gleichabständige Einheiten. Abstände und Verhältnisse sind interpretierbar.

\textbf{Beispiel:} Variable \texttt{hhinc} – Haushaltsnettoeinkommen in Euro. Die Differenz 
zwischen 2000€ und 3000€ ist genauso groß wie zwischen 4000€ und 5000€.

\textbf{Zulässige Statistiken:}
\begin{itemize}
    \item Alle Statistiken der vorherigen Niveaus
    \item Mittelwert
    \item Standardabweichung, Varianz
    \item Pearson-Korrelation
    \item Regression, t-Tests, ANOVA
\end{itemize}

\subsection{Aufgabe 1c: Univariate Analyse von \texttt{ep04} (ordinal)}

\subsubsection{Häufigkeitsverteilung}

% TODO: Tabelle aus Python einfügen
% \input{tables/ep04_frequencies.tex}

\textbf{Interpretation:} [TODO: Beschreibung der Verteilung]

\subsubsection{Grafische Darstellung}

\begin{figure}[h]
    \centering
    % TODO: Abbildung aus Python einfügen
    % \includegraphics[width=0.85\textwidth]{../figures/block1/ep04_distribution.pdf}
    \caption{Häufigkeitsverteilung der Variable \texttt{ep04} – Erwartete Wirtschaftslage in 1 Jahr.}
    \source{Eigene Darstellung auf Basis ALLBUS 2021.}
    \label{fig:ep04}
\end{figure}

\subsubsection{Lagemaße}

\begin{itemize}
    \item \textbf{Modus:} [TODO: Häufigste Kategorie]
    \item \textbf{Median:} [TODO: Mittlere Kategorie]
\end{itemize}

\textbf{Interpretation:} [TODO: Interpretation der Lagemaße]

\subsection{Aufgabe 1d: Univariate Analyse von \texttt{hhinc} (metrisch)}

\subsubsection{Klassenbildung}

Für die Darstellung der Einkommensverteilung wurden folgende Klassen gebildet:

% TODO: Tabelle der Klassenbildung einfügen
% \input{tables/hhinc_classes.tex}

\subsubsection{Histogramm}

\begin{figure}[h]
    \centering
    % TODO: Histogramm aus Python einfügen
    % \includegraphics[width=0.85\textwidth]{../figures/block1/hhinc_histogram.pdf}
    \caption{Histogramm des Haushaltsnettoeinkommens (\texttt{hhinc}).}
    \source{Eigene Darstellung auf Basis ALLBUS 2021.}
    \label{fig:hhinc}
\end{figure}

\textbf{Häufigkeitsdichte vs. absolute Häufigkeit:} Das Histogramm zeigt die Häufigkeitsdichte, 
sodass die Fläche jedes Balkens proportional zur relativen Häufigkeit der entsprechenden 
Klasse ist. Dies ist bei unterschiedlich breiten Klassen wichtig für eine korrekte Interpretation.

\subsubsection{Statistische Kennwerte}

% TODO: Kennwerte-Tabelle aus Python einfügen
% \input{tables/hhinc_statistics.tex}

\begin{itemize}
    \item \textbf{Mittelwert:} [TODO] Euro
    \item \textbf{Standardabweichung:} [TODO] Euro
    \item \textbf{Median:} [TODO] Euro
    \item \textbf{Minimum:} [TODO] Euro
    \item \textbf{Maximum:} [TODO] Euro
\end{itemize}

\textbf{Interpretation:} [TODO: Interpretation der Kennwerte, Schiefe der Verteilung, 
Vergleich Mittelwert vs. Median]

    % ============================================================
%          BLOCK 2: BIVARIATE DESKRIPTION
% ============================================================

\section{Bivariate Deskription}
\label{sec:block2}

\subsection{Aufgabe 2a: Zusammenhang \texttt{ep01} × \texttt{fe14} (ordinal)}

\subsubsection{Kreuztabelle}

% TODO: Kreuztabelle aus Python einfügen
% \input{tables/ep01_fe14_crosstab.tex}

\textbf{Interpretation:} [TODO: Beschreibung der gemeinsamen Verteilung]

\subsubsection{Grafische Darstellung}

\begin{figure}[h]
    \centering
    % TODO: Mosaikdiagramm aus Python einfügen
    % \includegraphics[width=0.85\textwidth]{../figures/block2/ep01_fe14_mosaic.pdf}
    \caption{Mosaikdiagramm: Zusammenhang zwischen aktueller Wirtschaftseinschätzung 
    (\texttt{ep01}) und Erziehungsziel „beliebt sein" (\texttt{fe14}).}
    \source{Eigene Darstellung auf Basis ALLBUS 2021.}
    \label{fig:ep01_fe14}
\end{figure}

\subsubsection{Zusammenhangsmaß}

Für den Zusammenhang zweier ordinaler Variablen eignet sich die Spearman-Rangkorrelation:

\begin{itemize}
    \item \textbf{Spearman-Korrelation:} $r_s =$ [TODO]
    \item \textbf{p-Wert:} [TODO]
\end{itemize}

\textbf{Interpretation:} [TODO: Stärke und Richtung des Zusammenhangs, Signifikanz]

\subsection{Aufgabe 2b: Korrelation \texttt{xt10} und \texttt{age} (metrisch)}

\subsubsection{Streudiagramm}

\begin{figure}[h]
    \centering
    % TODO: Streudiagramm aus Python einfügen
    % \includegraphics[width=0.85\textwidth]{../figures/block2/xt10_age_scatter.pdf}
    \caption{Streudiagramm: Zusammenhang zwischen Alter (\texttt{age}) und 
    Interviewdauer (\texttt{xt10}).}
    \source{Eigene Darstellung auf Basis ALLBUS 2021.}
    \label{fig:xt10_age}
\end{figure}

\subsubsection{Pearson-Korrelation}

% TODO: Korrelationstabelle aus Python einfügen
% \input{tables/xt10_age_correlation.tex}

\begin{itemize}
    \item \textbf{Pearson-Korrelation:} $r =$ [TODO]
    \item \textbf{p-Wert:} [TODO]
    \item \textbf{Stichprobengröße:} $n =$ [TODO]
\end{itemize}

\textbf{Interpretation:} [TODO: Stärke und Richtung des linearen Zusammenhangs, 
statistische Signifikanz]

\subsection{Aufgabe 2c: Korrelation ist nicht Kausalität}

Die gefundene Korrelation zwischen Alter und Interviewdauer [TODO: positiv/negativ/nicht signifikant] 
bedeutet \textbf{nicht}, dass das Alter die Interviewdauer verursacht (oder umgekehrt).

\subsubsection{Mögliche alternative Erklärungen}

\begin{enumerate}
    \item \textbf{Drittvariablen:} Eine nicht beobachtete Variable könnte beide beeinflussen. 
    Beispielsweise könnte Bildungsniveau sowohl mit Alter als auch mit der Ausführlichkeit 
    der Antworten korrelieren.
    
    \item \textbf{Interviewer-Effekte:} Die Interviewdauer hängt stark vom Interviewer ab 
    (Fragetempo, Gesprächsführung). Wenn bestimmte Interviewer bevorzugt ältere oder jüngere 
    Personen befragen, entsteht eine Scheinkorrelation.
    
    \item \textbf{Umgekehrte Kausalität:} Auch wenn eine kausale Beziehung bestünde, ist unklar, 
    in welche Richtung sie verläuft.
    
    \item \textbf{Zufälligkeit:} Bei einem Signifikanzniveau von 5\% ist jede 20. Korrelation 
    zufällig signifikant, selbst wenn kein echter Zusammenhang besteht.
\end{enumerate}

\subsubsection{Bedingungen für Kausalität}

Um eine kausale Aussage zu rechtfertigen, wären erforderlich:

\begin{itemize}
    \item \textbf{Zeitliche Reihenfolge:} Die Ursache muss der Wirkung vorausgehen
    \item \textbf{Ausschluss von Drittvariablen:} Kontrolle konfundierender Faktoren
    \item \textbf{Experimentelles Design:} Randomisierte Kontrollstudien (hier nicht möglich)
    \item \textbf{Theoretische Plausibilität:} Mechanismus muss erklärbar sein
\end{itemize}

\textbf{Fazit:} Beobachtete Korrelationen in Querschnittsdaten sind wichtige Hinweise auf 
Zusammenhänge, erlauben aber \textit{per se} keine kausalen Schlussfolgerungen. Weiterführende 
Analysen (z.\,B. Längsschnittstudien, Experimente) wären nötig, um Kausalität zu belegen.

    % ============================================================
%          BLOCK 3: INFERENZSTATISTIK
% ============================================================

\section{Inferenzstatistik mit \texttt{lm02}}
\label{sec:block3}

\subsection{Aufgabe 3a: Grundgesamtheit vs. Stichprobe}

\subsubsection{Grundgesamtheit (Population)}

Die \textbf{Grundgesamtheit} im Kontext des ALLBUS 2021 umfasst die deutschsprachige 
Wohnbevölkerung in Privathaushalten ab 18 Jahren in Deutschland. Dies entspricht etwa 
70 Millionen Personen.

\subsubsection{Stichprobe}

Die \textbf{Stichprobe} besteht aus den tatsächlich befragten [TODO: N] Personen, die nach 
einem Zufallsverfahren ausgewählt wurden. Diese Stichprobe soll repräsentativ für die 
Grundgesamtheit sein.

\subsubsection{Inferenzstatistischer Ansatz}

Die Inferenzstatistik erlaubt es, von den Stichprobendaten auf die Grundgesamtheit zu 
schließen:

\begin{itemize}
    \item \textbf{Punktschätzung:} Der Stichprobenmittelwert $\bar{x}$ schätzt den 
    Populationsmittelwert $\mu$
    \item \textbf{Intervallschätzung:} Konfidenzintervalle geben einen Bereich an, in dem 
    $\mu$ mit bestimmter Wahrscheinlichkeit liegt
    \item \textbf{Hypothesentests:} Prüfen von Vermutungen über die Population
\end{itemize}

\subsection{Aufgabe 3b: Konfidenzintervall für \texttt{lm02}}

Variable \texttt{lm02} erfasst die tägliche Fernsehdauer in Minuten.

\subsubsection{Deskriptive Statistik}

% TODO: Statistik-Tabelle aus Python einfügen
% \input{tables/lm02_descriptives.tex}

\begin{itemize}
    \item \textbf{Stichprobenmittelwert:} $\bar{x} =$ [TODO] Minuten
    \item \textbf{Standardabweichung:} $s =$ [TODO] Minuten
    \item \textbf{Stichprobengröße:} $n =$ [TODO]
\end{itemize}

\subsubsection{95\%-Konfidenzintervall}

Das Konfidenzintervall wurde mit der t-Verteilung berechnet (da $\sigma$ unbekannt):

\begin{equation}
    \text{KI}_{95\%} = \bar{x} \pm t_{n-1; 0{,}975} \cdot \frac{s}{\sqrt{n}}
\end{equation}

\textbf{Ergebnis:} [TODO: untere Grenze] $< \mu <$ [TODO: obere Grenze] Minuten

\subsubsection{Interpretation}

Das 95\%-Konfidenzintervall bedeutet: Wenn wir das Stichprobenverfahren unendlich oft 
wiederholen würden, lägen in 95\% der Fälle die so konstruierten Intervalle den wahren 
Populationsmittelwert $\mu$ enthalten.

\textbf{Wichtig:} Es bedeutet \textit{nicht}, dass der wahre Wert mit 95\% Wahrscheinlichkeit 
in diesem spezifischen Intervall liegt (der wahre Wert ist fix, nur unbekannt).

\subsection{Aufgabe 3c: t-Test für zwei unabhängige Gruppen}

\subsubsection{Fragestellung}

Unterscheidet sich die durchschnittliche Fernsehdauer (\texttt{lm02}) zwischen Männern 
und Frauen (\texttt{sex})?

\subsubsection{Hypothesen}

\begin{itemize}
    \item $H_0$: $\mu_{\text{männlich}} = \mu_{\text{weiblich}}$ 
    (kein Unterschied in der mittleren Fernsehdauer)
    \item $H_1$: $\mu_{\text{männlich}} \neq \mu_{\text{weiblich}}$ 
    (Unterschied besteht, zweiseitig)
\end{itemize}

Signifikanzniveau: $\alpha = 0{,}05$

\subsubsection{Deskriptive Statistik nach Gruppen}

% TODO: Gruppentabelle aus Python einfügen
% \input{tables/lm02_by_sex.tex}

\subsubsection{Testergebnis}

% TODO: t-Test-Ergebnis aus Python einfügen
% \input{tables/lm02_ttest.tex}

\begin{itemize}
    \item \textbf{t-Statistik:} $t =$ [TODO]
    \item \textbf{Freiheitsgrade:} $df =$ [TODO]
    \item \textbf{p-Wert:} $p =$ [TODO]
\end{itemize}

\subsubsection{Interpretation}

[TODO: Entscheidung über H₀]

\textbf{Bei $p < 0{,}05$:} Wir verwerfen $H_0$ und schließen, dass ein statistisch 
signifikanter Unterschied in der mittleren Fernsehdauer zwischen den Geschlechtern besteht.

\textbf{Bei $p \geq 0{,}05$:} Wir können $H_0$ nicht verwerfen. Die Daten liefern keine 
ausreichende Evidenz für einen Unterschied.

\textbf{Praktische Relevanz:} [TODO: Größe des Unterschieds bewerten]

    % ============================================================
%          BLOCK 4: KORRELATION UND REGRESSION
% ============================================================

\section{Korrelation und Regression: \texttt{age} und \texttt{hhinc}}
\label{sec:block4}

\subsection{Aufgabe 4a: Pearson-Korrelation und Signifikanztest}

\subsubsection{Fragestellung}

Besteht ein linearer Zusammenhang zwischen Alter (\texttt{age}) und Haushaltsnettoeinkommen 
(\texttt{hhinc})?

\subsubsection{Korrelationskoeffizient}

% TODO: Korrelationstabelle aus Python einfügen
% \input{tables/age_hhinc_correlation.tex}

\begin{itemize}
    \item \textbf{Pearson-Korrelation:} $r =$ [TODO]
    \item \textbf{p-Wert:} $p =$ [TODO]
    \item \textbf{Stichprobengröße:} $n =$ [TODO]
\end{itemize}

\subsubsection{Interpretation}

\textbf{Stärke:} [TODO: schwach/moderat/stark basierend auf $|r|$]

\textbf{Richtung:} [TODO: positiv/negativ]

\textbf{Signifikanz:} Bei $\alpha = 0{,}05$ ist die Korrelation [TODO: signifikant/nicht signifikant].

\subsection{Aufgabe 4b: Streudiagramm}

\begin{figure}[h]
    \centering
    % TODO: Streudiagramm aus Python einfügen
    % \includegraphics[width=0.85\textwidth]{../figures/block4/age_hhinc_scatter.pdf}
    \caption{Streudiagramm: Zusammenhang zwischen Alter (\texttt{age}) und 
    Haushaltsnettoeinkommen (\texttt{hhinc}).}
    \source{Eigene Darstellung auf Basis ALLBUS 2021.}
    \label{fig:age_hhinc}
\end{figure}

\textbf{Visuelle Beurteilung:} [TODO: Linearität, Ausreißer, Streuung beschreiben]

\subsection{Aufgabe 4c: Einfache lineare Regression}

\subsubsection{Modellspezifikation}

Wir schätzen das lineare Modell:

\begin{equation}
    \text{hhinc}_i = \beta_0 + \beta_1 \cdot \text{age}_i + \varepsilon_i
\end{equation}

wobei:
\begin{itemize}
    \item $\beta_0$: Achsenabschnitt (erwartetes Einkommen bei Alter = 0)
    \item $\beta_1$: Steigung (Änderung des Einkommens pro Jahr Alter)
    \item $\varepsilon_i$: Fehlerterm
\end{itemize}

\subsubsection{Schätzergebnisse}

% TODO: Regressionstabelle aus Python einfügen
% \input{tables/age_hhinc_regression.tex}

\begin{itemize}
    \item \textbf{Intercept} ($\hat{\beta}_0$): [TODO] Euro
    \item \textbf{Steigung} ($\hat{\beta}_1$): [TODO] Euro pro Jahr
    \item \textbf{R²:} [TODO]
    \item \textbf{Adjustiertes R²:} [TODO]
    \item \textbf{F-Statistik:} $F =$ [TODO], $p =$ [TODO]
\end{itemize}

\subsubsection{Interpretation der Koeffizienten}

\textbf{Intercept:} [TODO: Interpretation – meist nicht sinnvoll, da Alter = 0 außerhalb 
des Datenbereichs]

\textbf{Steigung:} Pro zusätzlichem Lebensjahr ändert sich das erwartete Haushaltsnettoeinkommen 
um [TODO] Euro. [TODO: Vorzeichen und praktische Bedeutung diskutieren]

\textbf{R²:} Das Modell erklärt [TODO]\% der Varianz im Haushaltsnettoeinkommen durch das Alter.

\subsubsection{Grafische Darstellung mit Regressionsgerade}

\begin{figure}[h]
    \centering
    % TODO: Streudiagramm mit Regressionslinie aus Python einfügen
    % \includegraphics[width=0.85\textwidth]{../figures/block4/age_hhinc_regression.pdf}
    \caption{Streudiagramm mit geschätzter Regressionsgerade.}
    \source{Eigene Darstellung auf Basis ALLBUS 2021.}
    \label{fig:age_hhinc_reg}
\end{figure}

\subsubsection{Beispielprognose}

Für eine Person im Alter von 45 Jahren lautet die Prognose:

\begin{equation}
    \widehat{\text{hhinc}}_{45} = \hat{\beta}_0 + \hat{\beta}_1 \cdot 45 = \text{[TODO]} \text{ Euro}
\end{equation}

\textbf{Einschränkung:} Diese Prognose ist nur innerhalb des beobachteten Altersbereichs 
sinnvoll (keine Extrapolation).

\subsubsection{Modellannahmen und Limitationen}

Das einfache lineare Modell unterstellt:
\begin{itemize}
    \item Lineare Beziehung zwischen Alter und Einkommen
    \item Homoskedastizität (konstante Fehlervarianz)
    \item Normalverteilte Residuen
    \item Keine Ausreißer mit starkem Einfluss
\end{itemize}

[TODO: Kurze Diskussion, ob diese Annahmen erfüllt sind oder Einschränkungen bestehen]

    % ============================================================
%          BLOCK 5: EINFAKTORIELLE ANOVA
% ============================================================

\section{Einfaktorielle ANOVA: \texttt{gd02} nach \texttt{hs01}}
\label{sec:block5}

\subsection{Aufgabe 5a: Gruppenbildung und deskriptive Statistik}

\subsubsection{Gruppierungsvariable}

Variable \texttt{hs01} erfasst den subjektiven Gesundheitszustand mit [TODO: Anzahl] 
Kategorien (z.\,B. „sehr gut", „gut", „zufriedenstellend", „weniger gut", „schlecht").

\subsubsection{Zielvariable}

Variable \texttt{gd02} misst die Wohndauer im aktuellen Wohnort in Jahren. Personen mit 
Wohndauer „unter 1 Jahr" wurden als 0 Jahre kodiert.

\subsubsection{Deskriptive Statistik nach Gruppen}

% TODO: Gruppentabelle aus Python einfügen
% \input{tables/gd02_by_hs01_descriptives.tex}

Die Tabelle zeigt für jede Gesundheitsgruppe:
\begin{itemize}
    \item Anzahl der Beobachtungen ($n$)
    \item Mittelwert der Wohndauer ($\bar{x}$)
    \item Standardabweichung ($s$)
\end{itemize}

\textbf{Erste Eindrücke:} [TODO: Unterschiede zwischen Gruppen beschreiben]

\subsection{Aufgabe 5b: Unabhängigkeit der Gruppen}

\subsubsection{Voraussetzung für ANOVA}

Die einfaktorielle ANOVA setzt voraus, dass die Beobachtungen in den Gruppen unabhängig sind 
(d.\,h. keine Person gehört zu mehreren Gruppen).

\subsubsection{Argumentation}

Im ALLBUS-Datensatz gilt:
\begin{itemize}
    \item Jede Person wurde nur einmal befragt
    \item Der Gesundheitszustand (\texttt{hs01}) ist ein Personenmerkmal – jede Person fällt 
    in genau eine Kategorie
    \item Es gibt keine offensichtlichen Cluster (z.\,B. Haushalte mit mehreren Befragten)
    \item Die Stichprobenziehung erfolgte unabhängig
\end{itemize}

\textbf{Fazit:} Die Unabhängigkeitsannahme ist plausibel erfüllt.

\subsection{Aufgabe 5c: Einfaktorielle ANOVA bei $\alpha = 0{,}20$}

\subsubsection{Hypothesen}

\begin{itemize}
    \item $H_0$: $\mu_1 = \mu_2 = \ldots = \mu_k$ (alle Gruppen haben dieselbe mittlere Wohndauer)
    \item $H_1$: Mindestens zwei Gruppenmittelwerte unterscheiden sich
\end{itemize}

Signifikanzniveau: $\alpha = 0{,}20$ (ungewöhnlich hoch, aber laut Aufgabenstellung vorgegeben)

\subsubsection{ANOVA-Tabelle}

% TODO: ANOVA-Tabelle aus Python einfügen
% \input{tables/gd02_anova.tex}

\begin{itemize}
    \item \textbf{F-Statistik:} $F =$ [TODO]
    \item \textbf{Freiheitsgrade:} $df_{\text{between}} =$ [TODO], $df_{\text{within}} =$ [TODO]
    \item \textbf{p-Wert:} $p =$ [TODO]
\end{itemize}

\subsubsection{Interpretation}

\textbf{Bei $p < 0{,}20$:} Wir verwerfen $H_0$ und schließen, dass sich die mittlere Wohndauer 
zwischen mindestens zwei Gesundheitsgruppen signifikant unterscheidet.

\textbf{Bei $p \geq 0{,}20$:} Die Daten liefern keine ausreichende Evidenz für Unterschiede.

\subsubsection{Post-Hoc-Tests}

Da die ANOVA nur feststellt, \textit{dass} Unterschiede bestehen (nicht \textit{welche}), 
werden Post-Hoc-Tests (z.\,B. Tukey HSD) durchgeführt:

% TODO: Post-Hoc-Ergebnis aus Python einfügen
% \input{tables/gd02_posthoc.tex}

\textbf{Signifikante Paarvergleiche:} [TODO: Welche Gruppen unterscheiden sich?]

\subsection{Diskussion des Signifikanzniveaus}

\subsubsection{Einfluss von $\alpha$ auf die Ergebnisse}

Das gewählte $\alpha = 0{,}20$ ist deutlich höher als das übliche 5\%-Niveau:

\begin{itemize}
    \item \textbf{Bei $\alpha = 0{,}05$:} [TODO: Wären die Ergebnisse noch signifikant?]
    \item \textbf{Bei $\alpha = 0{,}20$:} Höhere Wahrscheinlichkeit, $H_0$ zu verwerfen 
    (auch bei kleinen Effekten)
\end{itemize}

\subsubsection{Typ-I- vs. Typ-II-Fehler}

\begin{itemize}
    \item \textbf{Typ-I-Fehler} ($\alpha$): Fälschliche Ablehnung von $H_0$ (falsch positiv)
    \item \textbf{Typ-II-Fehler} ($\beta$): Fälschliches Beibehalten von $H_0$ (falsch negativ)
\end{itemize}

Ein höheres $\alpha$ erhöht die Wahrscheinlichkeit eines Typ-I-Fehlers, reduziert aber 
gleichzeitig die Wahrscheinlichkeit eines Typ-II-Fehlers (höhere Power).

\subsubsection{Praktische Implikationen}

In explorativen Studien kann ein höheres $\alpha$ gerechtfertigt sein, um keine potenziell 
wichtigen Effekte zu übersehen. In konfirmatorischen Studien (z.\,B. klinische Tests) ist 
hingegen ein strengeres Niveau erforderlich.

\textbf{Fazit:} Die Wahl von $\alpha = 0{,}20$ führt dazu, dass [TODO: mehr/weniger] Gruppen-
unterschiede als signifikant identifiziert werden. Dies sollte bei der Interpretation 
berücksichtigt werden.

    % ============================================================
%                       DISKUSSION
% ============================================================

\section{Diskussion}
\label{sec:diskussion}

\subsection{Zusammenfassung der Hauptergebnisse}

Dieses Workbook wendete deskriptive und inferenzstatistische Methoden auf den ALLBUS 2021 
Teildatensatz an. Die wichtigsten Ergebnisse:

\begin{itemize}
    \item \textbf{Block 1:} [TODO: Kernaussagen zur univariaten Analyse]
    \item \textbf{Block 2:} [TODO: Kernaussagen zu bivariaten Zusammenhängen]
    \item \textbf{Block 3:} [TODO: Kernaussagen zu Konfidenzintervall und t-Test]
    \item \textbf{Block 4:} [TODO: Kernaussagen zur Regression]
    \item \textbf{Block 5:} [TODO: Kernaussagen zur ANOVA]
\end{itemize}

\subsection{Methodische Überlegungen}

\subsubsection{Stärken des Vorgehens}

\begin{itemize}
    \item Systematische Anwendung skalenniveaugerechter Methoden
    \item Verwendung etablierter statistischer Tests mit klaren Voraussetzungen
    \item Transparente Dokumentation aller Analyseschritte
    \item Reproduzierbarkeit durch Python-Skripte
\end{itemize}

\subsubsection{Limitationen}

\begin{itemize}
    \item \textbf{Querschnittsdaten:} Keine kausalen Aussagen möglich (nur Korrelationen)
    \item \textbf{Fehlende Werte:} Bereinigung könnte systematische Verzerrungen einführen
    \item \textbf{Einfache Modelle:} Regression mit nur einem Prädiktor ignoriert konfundierende 
    Variablen
    \item \textbf{Annahmenverletzungen:} Nicht alle Voraussetzungen (z.\,B. Normalverteilung) 
    wurden formal getestet
\end{itemize}

\subsection{Interpretation im Kontext}

\subsubsection{Plausibilität der Ergebnisse}

[TODO: Sind die gefundenen Zusammenhänge inhaltlich plausibel? Decken sie sich mit 
Erwartungen oder bisheriger Forschung?]

\subsubsection{Statistische vs. praktische Signifikanz}

Ein statistisch signifikantes Ergebnis bedeutet nicht automatisch praktische Relevanz. 
Bei großen Stichproben (wie im ALLBUS) werden auch sehr kleine Effekte signifikant.

[TODO: Beispiel: Ist der gefundene Unterschied/Zusammenhang groß genug, um praktisch 
bedeutsam zu sein?]

\subsection{Weiterführende Analysen}

Aufbauend auf diesen Ergebnissen könnten folgende Analysen sinnvoll sein:

\begin{itemize}
    \item \textbf{Multiple Regression:} Einbezug mehrerer Prädiktoren zur Kontrolle von 
    Drittvariablen
    \item \textbf{Interaktionseffekte:} Prüfen, ob Zusammenhänge in verschiedenen Subgruppen 
    unterschiedlich stark sind
    \item \textbf{Robustheitschecks:} Nicht-parametrische Tests bei Verletzung von Annahmen
    \item \textbf{Vergleich mit anderen ALLBUS-Wellen:} Zeitliche Entwicklung untersuchen
\end{itemize}

    % ============================================================
%                         FAZIT
% ============================================================

\section{Fazit}
\label{sec:fazit}

\subsection{Zentrale Erkenntnisse}

Diese Arbeit demonstrierte die praktische Anwendung grundlegender statistischer Methoden 
auf reale Umfragedaten. Die wichtigsten Erkenntnisse:

\begin{enumerate}
    \item \textbf{Skalenniveaus entscheiden über Methoden:} Die korrekte Identifikation von 
    nominal, ordinal und metrisch ist essentiell für die Wahl geeigneter Analysen.
    
    \item \textbf{Deskription vor Inferenz:} Eine gründliche deskriptive Analyse (Häufigkeiten, 
    Verteilungen, Grafiken) ist die Grundlage für alle weiterführenden Tests.
    
    \item \textbf{Korrelation ≠ Kausalität:} Beobachtete Zusammenhänge in Querschnittsdaten 
    erlauben keine kausalen Schlüsse ohne zusätzliche theoretische und methodische Absicherung.
    
    \item \textbf{Signifikanz vs. Relevanz:} Statistische Signifikanz und praktische Bedeutung 
    sind zu unterscheiden – besonders bei großen Stichproben.
    
    \item \textbf{Reproduzierbarkeit durch Code:} Die Verwendung von Python-Skripten ermöglicht 
    transparente und nachvollziehbare Analysen.
\end{enumerate}

\subsection{Methodische Reflexion}

Die Bearbeitung dieses Workbooks verdeutlichte:

\begin{itemize}
    \item Die Bedeutung sorgfältiger Datenbereinigung (Umgang mit fehlenden Werten, Sondercodes)
    \item Die Notwendigkeit, Voraussetzungen statistischer Tests zu prüfen
    \item Den Wert grafischer Darstellungen für das Verständnis von Daten
    \item Die Grenzen einfacher statistischer Modelle bei komplexen sozialen Phänomenen
\end{itemize}

\subsection{Ausblick}

Weiterführende Analysen könnten:

\begin{itemize}
    \item Multiple Prädiktoren in Regressionsmodellen berücksichtigen
    \item Interaktionseffekte zwischen Variablen untersuchen
    \item Zeitliche Trends durch Vergleich mit früheren ALLBUS-Wellen analysieren
    \item Fortgeschrittene Methoden (z.\,B. logistische Regression, Strukturgleichungsmodelle) 
    anwenden
\end{itemize}

\subsection{Abschließende Bemerkung}

Der ALLBUS-Datensatz bietet eine wertvolle Ressource für empirische sozialwissenschaftliche 
Forschung. Die hier durchgeführten Analysen kratzen nur an der Oberfläche dessen, was mit 
diesen Daten möglich ist. Sie demonstrieren jedoch grundlegende statistische Kompetenzen, 
die für weiterführende Analysen unerlässlich sind.


    % ---------- Literaturverzeichnis ----------
    \clearpage
    % \printbibliography[title={Literaturverzeichnis}]

    % ---------- Anhänge (bei Bedarf) ----------
    % \clearpage
    % \appendix
    % \section{Zusätzliche Abbildungen}
    % \section{Python-Code}

\end{document}
