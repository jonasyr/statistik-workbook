% ============================================================
%                    DATENBESCHREIBUNG
% ============================================================

\section{Datenbeschreibung}
\label{sec:daten}

\subsection{Der ALLBUS 2021 Teildatensatz}

Der verwendete Datensatz ist ein Auszug aus dem ALLBUS 2021 (ZA5284, Version 1.1.0) und 
enthält 29 Variablen aus verschiedenen Themenbereichen:

\begin{itemize}
    \item Soziodemografische Merkmale (Alter, Geschlecht, Bildung, Einkommen)
    \item Politische und wirtschaftliche Einschätzungen
    \item Werthaltungen und Einstellungen
    \item Gesundheit und Lebenssituation
\end{itemize}

Die Stichprobe umfasst \texttt{N = [ANZAHL]} Befragte aus der deutschsprachigen 
Wohnbevölkerung ab 18 Jahren.

\subsection{Verwendete Variablen}

Gemäß der individuellen Variablenzuteilung werden in dieser Arbeit folgende Variablen analysiert:

\subsubsection{Block 1: Univariate Deskription}

\begin{itemize}
    \item \textbf{mc04} (nominal): Ausländer im Freundeskreis
    \item \textbf{ep04} (ordinal): Wirtschaftslage in Deutschland in 1 Jahr
    \item \textbf{hhinc} (metrisch): Haushaltsnettoeinkommen in Euro
\end{itemize}

\subsubsection{Block 2: Bivariate Deskription}

\begin{itemize}
    \item \textbf{ep01} (ordinal): Aktuelle Wirtschaftslage in Deutschland
    \item \textbf{fe14} (ordinal): Erziehungsziel „beliebt sein"
    \item \textbf{xt10} (metrisch): Interviewdauer in Minuten
    \item \textbf{age} (metrisch): Alter in Jahren
\end{itemize}

\subsubsection{Block 3: Inferenzstatistik}

\begin{itemize}
    \item \textbf{lm02} (metrisch): Tägliche Fernsehdauer in Minuten
    \item \textbf{sex} (nominal): Geschlecht (für Gruppenvergleich)
\end{itemize}

\subsubsection{Block 4: Regression}

\begin{itemize}
    \item \textbf{age} (metrisch): Alter in Jahren (Prädiktor)
    \item \textbf{hhinc} (metrisch): Haushaltsnettoeinkommen (Zielvariable)
\end{itemize}

\subsubsection{Block 5: ANOVA}

\begin{itemize}
    \item \textbf{hs01} (ordinal): Gesundheitszustand (Gruppierungsfaktor)
    \item \textbf{gd02} (metrisch): Wohndauer im Ort in Jahren
\end{itemize}

\subsection{Datenbereinigung}

Vor den Analysen wurden folgende Bereinigungsschritte durchgeführt:

\begin{enumerate}
    \item Behandlung fehlender Werte (Sondercodes wie -99, -98, -42 etc.)
    \item Konvertierung von Datentypen (numerisch vs. kategorial)
    \item Plausibilitätsprüfungen (Wertebereich, Ausreißer)
    \item Speicherung als bereinigte CSV-Datei (\texttt{allbus\_clean.csv})
\end{enumerate}

Details zur Datenbereinigung finden sich im Python-Skript \texttt{src/data\_prep.py}.
