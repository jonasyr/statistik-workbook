% ============================================================
%          BLOCK 4: KORRELATION UND REGRESSION
% ============================================================

\section{Korrelation und Regression: \texttt{age} und \texttt{hhinc}}
\label{sec:block4}

\subsection{Aufgabe 4a: Pearson-Korrelation und Signifikanztest}

\subsubsection{Fragestellung}

Besteht ein linearer Zusammenhang zwischen Alter (\texttt{age}) und Haushaltsnettoeinkommen 
(\texttt{hhinc})?

\subsubsection{Korrelationskoeffizient}

% TODO: Korrelationstabelle aus Python einfügen
% \input{tables/age_hhinc_correlation.tex}

\begin{itemize}
    \item \textbf{Pearson-Korrelation:} $r =$ [TODO]
    \item \textbf{p-Wert:} $p =$ [TODO]
    \item \textbf{Stichprobengröße:} $n =$ [TODO]
\end{itemize}

\subsubsection{Interpretation}

\textbf{Stärke:} [TODO: schwach/moderat/stark basierend auf $|r|$]

\textbf{Richtung:} [TODO: positiv/negativ]

\textbf{Signifikanz:} Bei $\alpha = 0{,}05$ ist die Korrelation [TODO: signifikant/nicht signifikant].

\subsection{Aufgabe 4b: Streudiagramm}

\begin{figure}[h]
    \centering
    % TODO: Streudiagramm aus Python einfügen
    % \includegraphics[width=0.85\textwidth]{../figures/block4/age_hhinc_scatter.pdf}
    \caption{Streudiagramm: Zusammenhang zwischen Alter (\texttt{age}) und 
    Haushaltsnettoeinkommen (\texttt{hhinc}).}
    \source{Eigene Darstellung auf Basis ALLBUS 2021.}
    \label{fig:age_hhinc}
\end{figure}

\textbf{Visuelle Beurteilung:} [TODO: Linearität, Ausreißer, Streuung beschreiben]

\subsection{Aufgabe 4c: Einfache lineare Regression}

\subsubsection{Modellspezifikation}

Wir schätzen das lineare Modell:

\begin{equation}
    \text{hhinc}_i = \beta_0 + \beta_1 \cdot \text{age}_i + \varepsilon_i
\end{equation}

wobei:
\begin{itemize}
    \item $\beta_0$: Achsenabschnitt (erwartetes Einkommen bei Alter = 0)
    \item $\beta_1$: Steigung (Änderung des Einkommens pro Jahr Alter)
    \item $\varepsilon_i$: Fehlerterm
\end{itemize}

\subsubsection{Schätzergebnisse}

% TODO: Regressionstabelle aus Python einfügen
% \input{tables/age_hhinc_regression.tex}

\begin{itemize}
    \item \textbf{Intercept} ($\hat{\beta}_0$): [TODO] Euro
    \item \textbf{Steigung} ($\hat{\beta}_1$): [TODO] Euro pro Jahr
    \item \textbf{R²:} [TODO]
    \item \textbf{Adjustiertes R²:} [TODO]
    \item \textbf{F-Statistik:} $F =$ [TODO], $p =$ [TODO]
\end{itemize}

\subsubsection{Interpretation der Koeffizienten}

\textbf{Intercept:} [TODO: Interpretation – meist nicht sinnvoll, da Alter = 0 außerhalb 
des Datenbereichs]

\textbf{Steigung:} Pro zusätzlichem Lebensjahr ändert sich das erwartete Haushaltsnettoeinkommen 
um [TODO] Euro. [TODO: Vorzeichen und praktische Bedeutung diskutieren]

\textbf{R²:} Das Modell erklärt [TODO]\% der Varianz im Haushaltsnettoeinkommen durch das Alter.

\subsubsection{Grafische Darstellung mit Regressionsgerade}

\begin{figure}[h]
    \centering
    % TODO: Streudiagramm mit Regressionslinie aus Python einfügen
    % \includegraphics[width=0.85\textwidth]{../figures/block4/age_hhinc_regression.pdf}
    \caption{Streudiagramm mit geschätzter Regressionsgerade.}
    \source{Eigene Darstellung auf Basis ALLBUS 2021.}
    \label{fig:age_hhinc_reg}
\end{figure}

\subsubsection{Beispielprognose}

Für eine Person im Alter von 45 Jahren lautet die Prognose:

\begin{equation}
    \widehat{\text{hhinc}}_{45} = \hat{\beta}_0 + \hat{\beta}_1 \cdot 45 = \text{[TODO]} \text{ Euro}
\end{equation}

\textbf{Einschränkung:} Diese Prognose ist nur innerhalb des beobachteten Altersbereichs 
sinnvoll (keine Extrapolation).

\subsubsection{Modellannahmen und Limitationen}

Das einfache lineare Modell unterstellt:
\begin{itemize}
    \item Lineare Beziehung zwischen Alter und Einkommen
    \item Homoskedastizität (konstante Fehlervarianz)
    \item Normalverteilte Residuen
    \item Keine Ausreißer mit starkem Einfluss
\end{itemize}

[TODO: Kurze Diskussion, ob diese Annahmen erfüllt sind oder Einschränkungen bestehen]
