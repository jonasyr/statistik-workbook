% ============================================================
%          BLOCK 5: EINFAKTORIELLE ANOVA
% ============================================================

\section{Einfaktorielle ANOVA: \texttt{gd02} nach \texttt{hs01}}
\label{sec:block5}

\subsection{Aufgabe 5a: Gruppenbildung und deskriptive Statistik}

\subsubsection{Gruppierungsvariable}

Variable \texttt{hs01} erfasst den subjektiven Gesundheitszustand mit [TODO: Anzahl] 
Kategorien (z.\,B. „sehr gut", „gut", „zufriedenstellend", „weniger gut", „schlecht").

\subsubsection{Zielvariable}

Variable \texttt{gd02} misst die Wohndauer im aktuellen Wohnort in Jahren. Personen mit 
Wohndauer „unter 1 Jahr" wurden als 0 Jahre kodiert.

\subsubsection{Deskriptive Statistik nach Gruppen}

% TODO: Gruppentabelle aus Python einfügen
% \input{tables/gd02_by_hs01_descriptives.tex}

Die Tabelle zeigt für jede Gesundheitsgruppe:
\begin{itemize}
    \item Anzahl der Beobachtungen ($n$)
    \item Mittelwert der Wohndauer ($\bar{x}$)
    \item Standardabweichung ($s$)
\end{itemize}

\textbf{Erste Eindrücke:} [TODO: Unterschiede zwischen Gruppen beschreiben]

\subsection{Aufgabe 5b: Unabhängigkeit der Gruppen}

\subsubsection{Voraussetzung für ANOVA}

Die einfaktorielle ANOVA setzt voraus, dass die Beobachtungen in den Gruppen unabhängig sind 
(d.\,h. keine Person gehört zu mehreren Gruppen).

\subsubsection{Argumentation}

Im ALLBUS-Datensatz gilt:
\begin{itemize}
    \item Jede Person wurde nur einmal befragt
    \item Der Gesundheitszustand (\texttt{hs01}) ist ein Personenmerkmal – jede Person fällt 
    in genau eine Kategorie
    \item Es gibt keine offensichtlichen Cluster (z.\,B. Haushalte mit mehreren Befragten)
    \item Die Stichprobenziehung erfolgte unabhängig
\end{itemize}

\textbf{Fazit:} Die Unabhängigkeitsannahme ist plausibel erfüllt.

\subsection{Aufgabe 5c: Einfaktorielle ANOVA bei $\alpha = 0{,}20$}

\subsubsection{Hypothesen}

\begin{itemize}
    \item $H_0$: $\mu_1 = \mu_2 = \ldots = \mu_k$ (alle Gruppen haben dieselbe mittlere Wohndauer)
    \item $H_1$: Mindestens zwei Gruppenmittelwerte unterscheiden sich
\end{itemize}

Signifikanzniveau: $\alpha = 0{,}20$ (ungewöhnlich hoch, aber laut Aufgabenstellung vorgegeben)

\subsubsection{ANOVA-Tabelle}

% TODO: ANOVA-Tabelle aus Python einfügen
% \input{tables/gd02_anova.tex}

\begin{itemize}
    \item \textbf{F-Statistik:} $F =$ [TODO]
    \item \textbf{Freiheitsgrade:} $df_{\text{between}} =$ [TODO], $df_{\text{within}} =$ [TODO]
    \item \textbf{p-Wert:} $p =$ [TODO]
\end{itemize}

\subsubsection{Interpretation}

\textbf{Bei $p < 0{,}20$:} Wir verwerfen $H_0$ und schließen, dass sich die mittlere Wohndauer 
zwischen mindestens zwei Gesundheitsgruppen signifikant unterscheidet.

\textbf{Bei $p \geq 0{,}20$:} Die Daten liefern keine ausreichende Evidenz für Unterschiede.

\subsubsection{Post-Hoc-Tests}

Da die ANOVA nur feststellt, \textit{dass} Unterschiede bestehen (nicht \textit{welche}), 
werden Post-Hoc-Tests (z.\,B. Tukey HSD) durchgeführt:

% TODO: Post-Hoc-Ergebnis aus Python einfügen
% \input{tables/gd02_posthoc.tex}

\textbf{Signifikante Paarvergleiche:} [TODO: Welche Gruppen unterscheiden sich?]

\subsection{Diskussion des Signifikanzniveaus}

\subsubsection{Einfluss von $\alpha$ auf die Ergebnisse}

Das gewählte $\alpha = 0{,}20$ ist deutlich höher als das übliche 5\%-Niveau:

\begin{itemize}
    \item \textbf{Bei $\alpha = 0{,}05$:} [TODO: Wären die Ergebnisse noch signifikant?]
    \item \textbf{Bei $\alpha = 0{,}20$:} Höhere Wahrscheinlichkeit, $H_0$ zu verwerfen 
    (auch bei kleinen Effekten)
\end{itemize}

\subsubsection{Typ-I- vs. Typ-II-Fehler}

\begin{itemize}
    \item \textbf{Typ-I-Fehler} ($\alpha$): Fälschliche Ablehnung von $H_0$ (falsch positiv)
    \item \textbf{Typ-II-Fehler} ($\beta$): Fälschliches Beibehalten von $H_0$ (falsch negativ)
\end{itemize}

Ein höheres $\alpha$ erhöht die Wahrscheinlichkeit eines Typ-I-Fehlers, reduziert aber 
gleichzeitig die Wahrscheinlichkeit eines Typ-II-Fehlers (höhere Power).

\subsubsection{Praktische Implikationen}

In explorativen Studien kann ein höheres $\alpha$ gerechtfertigt sein, um keine potenziell 
wichtigen Effekte zu übersehen. In konfirmatorischen Studien (z.\,B. klinische Tests) ist 
hingegen ein strengeres Niveau erforderlich.

\textbf{Fazit:} Die Wahl von $\alpha = 0{,}20$ führt dazu, dass [TODO: mehr/weniger] Gruppen-
unterschiede als signifikant identifiziert werden. Dies sollte bei der Interpretation 
berücksichtigt werden.
