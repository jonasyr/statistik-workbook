% ============================================================
%          BLOCK 1: UNIVARIATE DESKRIPTION
% ============================================================

\section{Univariate Deskription und Skalenniveaus}
\label{sec:block1}

\subsection{Aufgabe 1b: Skalenniveaus}

Die Wahl geeigneter statistischer Methoden hängt wesentlich vom Skalenniveau der Variablen ab. 
Man unterscheidet drei grundlegende Skalenniveaus:

\subsubsection{Nominalskalenniveau}

\textbf{Definition:} Variablen mit Nominalskalenniveau besitzen Kategorien ohne natürliche Ordnung. 
Es kann nur Gleichheit oder Verschiedenheit festgestellt werden.

\textbf{Beispiel:} Variable \texttt{mc04} – „Haben Sie Ausländer im Freundeskreis?" mit den 
Kategorien „Ja", „Nein", „Weiß nicht". Diese Kategorien haben keine sinnvolle Reihenfolge.

\textbf{Zulässige Statistiken:}
\begin{itemize}
    \item Häufigkeiten (absolut und relativ)
    \item Modus (häufigste Kategorie)
    \item Balken- oder Kreisdiagramme
\end{itemize}

\subsubsection{Ordinalskalenniveau}

\textbf{Definition:} Variablen mit Ordinalskalenniveau besitzen geordnete Kategorien. 
Die Abstände zwischen den Kategorien sind jedoch nicht interpretierbar.

\textbf{Beispiel:} Variable \texttt{ep04} – „Wie wird sich die wirtschaftliche Lage in Deutschland 
in einem Jahr entwickeln?" mit Kategorien von „wesentlich besser" bis „wesentlich schlechter". 
Die Kategorien sind klar geordnet, aber der Abstand zwischen „besser" und „wesentlich besser" 
ist nicht quantifizierbar.

\textbf{Zulässige Statistiken:}
\begin{itemize}
    \item Alle Statistiken des Nominalniveaus
    \item Median (mittlere Kategorie)
    \item Perzentile
    \item Rangkorrelationen (Spearman, Kendall)
\end{itemize}

\subsubsection{Metrisches Skalenniveau}

\textbf{Definition:} Variablen mit metrischem Skalenniveau (Intervall- oder Verhältnisskala) 
besitzen gleichabständige Einheiten. Abstände und Verhältnisse sind interpretierbar.

\textbf{Beispiel:} Variable \texttt{hhinc} – Haushaltsnettoeinkommen in Euro. Die Differenz 
zwischen 2000€ und 3000€ ist genauso groß wie zwischen 4000€ und 5000€.

\textbf{Zulässige Statistiken:}
\begin{itemize}
    \item Alle Statistiken der vorherigen Niveaus
    \item Mittelwert
    \item Standardabweichung, Varianz
    \item Pearson-Korrelation
    \item Regression, t-Tests, ANOVA
\end{itemize}

\subsection{Aufgabe 1c: Univariate Analyse von \texttt{ep04} (ordinal)}

\subsubsection{Häufigkeitsverteilung}

% TODO: Tabelle aus Python einfügen
% \input{tables/ep04_frequencies.tex}

\textbf{Interpretation:} [TODO: Beschreibung der Verteilung]

\subsubsection{Grafische Darstellung}

\begin{figure}[h]
    \centering
    % TODO: Abbildung aus Python einfügen
    % \includegraphics[width=0.85\textwidth]{../figures/block1/ep04_distribution.pdf}
    \caption{Häufigkeitsverteilung der Variable \texttt{ep04} – Erwartete Wirtschaftslage in 1 Jahr.}
    \source{Eigene Darstellung auf Basis ALLBUS 2021.}
    \label{fig:ep04}
\end{figure}

\subsubsection{Lagemaße}

\begin{itemize}
    \item \textbf{Modus:} [TODO: Häufigste Kategorie]
    \item \textbf{Median:} [TODO: Mittlere Kategorie]
\end{itemize}

\textbf{Interpretation:} [TODO: Interpretation der Lagemaße]

\subsection{Aufgabe 1d: Univariate Analyse von \texttt{hhinc} (metrisch)}

\subsubsection{Klassenbildung}

Für die Darstellung der Einkommensverteilung wurden folgende Klassen gebildet:

% TODO: Tabelle der Klassenbildung einfügen
% \input{tables/hhinc_classes.tex}

\subsubsection{Histogramm}

\begin{figure}[h]
    \centering
    % TODO: Histogramm aus Python einfügen
    % \includegraphics[width=0.85\textwidth]{../figures/block1/hhinc_histogram.pdf}
    \caption{Histogramm des Haushaltsnettoeinkommens (\texttt{hhinc}).}
    \source{Eigene Darstellung auf Basis ALLBUS 2021.}
    \label{fig:hhinc}
\end{figure}

\textbf{Häufigkeitsdichte vs. absolute Häufigkeit:} Das Histogramm zeigt die Häufigkeitsdichte, 
sodass die Fläche jedes Balkens proportional zur relativen Häufigkeit der entsprechenden 
Klasse ist. Dies ist bei unterschiedlich breiten Klassen wichtig für eine korrekte Interpretation.

\subsubsection{Statistische Kennwerte}

% TODO: Kennwerte-Tabelle aus Python einfügen
% \input{tables/hhinc_statistics.tex}

\begin{itemize}
    \item \textbf{Mittelwert:} [TODO] Euro
    \item \textbf{Standardabweichung:} [TODO] Euro
    \item \textbf{Median:} [TODO] Euro
    \item \textbf{Minimum:} [TODO] Euro
    \item \textbf{Maximum:} [TODO] Euro
\end{itemize}

\textbf{Interpretation:} [TODO: Interpretation der Kennwerte, Schiefe der Verteilung, 
Vergleich Mittelwert vs. Median]
