% ============================================================
%                       DISKUSSION
% ============================================================

\section{Diskussion}
\label{sec:diskussion}

\subsection{Zusammenfassung der Hauptergebnisse}

Dieses Workbook wendete deskriptive und inferenzstatistische Methoden auf den ALLBUS 2021 
Teildatensatz an. Die wichtigsten Ergebnisse:

\begin{itemize}
    \item \textbf{Block 1:} [TODO: Kernaussagen zur univariaten Analyse]
    \item \textbf{Block 2:} [TODO: Kernaussagen zu bivariaten Zusammenhängen]
    \item \textbf{Block 3:} [TODO: Kernaussagen zu Konfidenzintervall und t-Test]
    \item \textbf{Block 4:} [TODO: Kernaussagen zur Regression]
    \item \textbf{Block 5:} [TODO: Kernaussagen zur ANOVA]
\end{itemize}

\subsection{Methodische Überlegungen}

\subsubsection{Stärken des Vorgehens}

\begin{itemize}
    \item Systematische Anwendung skalenniveaugerechter Methoden
    \item Verwendung etablierter statistischer Tests mit klaren Voraussetzungen
    \item Transparente Dokumentation aller Analyseschritte
    \item Reproduzierbarkeit durch Python-Skripte
\end{itemize}

\subsubsection{Limitationen}

\begin{itemize}
    \item \textbf{Querschnittsdaten:} Keine kausalen Aussagen möglich (nur Korrelationen)
    \item \textbf{Fehlende Werte:} Bereinigung könnte systematische Verzerrungen einführen
    \item \textbf{Einfache Modelle:} Regression mit nur einem Prädiktor ignoriert konfundierende 
    Variablen
    \item \textbf{Annahmenverletzungen:} Nicht alle Voraussetzungen (z.\,B. Normalverteilung) 
    wurden formal getestet
\end{itemize}

\subsection{Interpretation im Kontext}

\subsubsection{Plausibilität der Ergebnisse}

[TODO: Sind die gefundenen Zusammenhänge inhaltlich plausibel? Decken sie sich mit 
Erwartungen oder bisheriger Forschung?]

\subsubsection{Statistische vs. praktische Signifikanz}

Ein statistisch signifikantes Ergebnis bedeutet nicht automatisch praktische Relevanz. 
Bei großen Stichproben (wie im ALLBUS) werden auch sehr kleine Effekte signifikant.

[TODO: Beispiel: Ist der gefundene Unterschied/Zusammenhang groß genug, um praktisch 
bedeutsam zu sein?]

\subsection{Weiterführende Analysen}

Aufbauend auf diesen Ergebnissen könnten folgende Analysen sinnvoll sein:

\begin{itemize}
    \item \textbf{Multiple Regression:} Einbezug mehrerer Prädiktoren zur Kontrolle von 
    Drittvariablen
    \item \textbf{Interaktionseffekte:} Prüfen, ob Zusammenhänge in verschiedenen Subgruppen 
    unterschiedlich stark sind
    \item \textbf{Robustheitschecks:} Nicht-parametrische Tests bei Verletzung von Annahmen
    \item \textbf{Vergleich mit anderen ALLBUS-Wellen:} Zeitliche Entwicklung untersuchen
\end{itemize}
