% ============================================================
%                         FAZIT
% ============================================================

\section{Fazit}
\label{sec:fazit}

\subsection{Zentrale Erkenntnisse}

Diese Arbeit demonstrierte die praktische Anwendung grundlegender statistischer Methoden 
auf reale Umfragedaten. Die wichtigsten Erkenntnisse:

\begin{enumerate}
    \item \textbf{Skalenniveaus entscheiden über Methoden:} Die korrekte Identifikation von 
    nominal, ordinal und metrisch ist essentiell für die Wahl geeigneter Analysen.
    
    \item \textbf{Deskription vor Inferenz:} Eine gründliche deskriptive Analyse (Häufigkeiten, 
    Verteilungen, Grafiken) ist die Grundlage für alle weiterführenden Tests.
    
    \item \textbf{Korrelation ≠ Kausalität:} Beobachtete Zusammenhänge in Querschnittsdaten 
    erlauben keine kausalen Schlüsse ohne zusätzliche theoretische und methodische Absicherung.
    
    \item \textbf{Signifikanz vs. Relevanz:} Statistische Signifikanz und praktische Bedeutung 
    sind zu unterscheiden – besonders bei großen Stichproben.
    
    \item \textbf{Reproduzierbarkeit durch Code:} Die Verwendung von Python-Skripten ermöglicht 
    transparente und nachvollziehbare Analysen.
\end{enumerate}

\subsection{Methodische Reflexion}

Die Bearbeitung dieses Workbooks verdeutlichte:

\begin{itemize}
    \item Die Bedeutung sorgfältiger Datenbereinigung (Umgang mit fehlenden Werten, Sondercodes)
    \item Die Notwendigkeit, Voraussetzungen statistischer Tests zu prüfen
    \item Den Wert grafischer Darstellungen für das Verständnis von Daten
    \item Die Grenzen einfacher statistischer Modelle bei komplexen sozialen Phänomenen
\end{itemize}

\subsection{Ausblick}

Weiterführende Analysen könnten:

\begin{itemize}
    \item Multiple Prädiktoren in Regressionsmodellen berücksichtigen
    \item Interaktionseffekte zwischen Variablen untersuchen
    \item Zeitliche Trends durch Vergleich mit früheren ALLBUS-Wellen analysieren
    \item Fortgeschrittene Methoden (z.\,B. logistische Regression, Strukturgleichungsmodelle) 
    anwenden
\end{itemize}

\subsection{Abschließende Bemerkung}

Der ALLBUS-Datensatz bietet eine wertvolle Ressource für empirische sozialwissenschaftliche 
Forschung. Die hier durchgeführten Analysen kratzen nur an der Oberfläche dessen, was mit 
diesen Daten möglich ist. Sie demonstrieren jedoch grundlegende statistische Kompetenzen, 
die für weiterführende Analysen unerlässlich sind.
