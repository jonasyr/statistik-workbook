% ============================================================
%          BLOCK 2: BIVARIATE DESKRIPTION
% ============================================================

\section{Bivariate Deskription}
\label{sec:block2}

\subsection{Aufgabe 2a: Zusammenhang \texttt{ep01} × \texttt{fe14} (ordinal)}

\subsubsection{Kreuztabelle}

% TODO: Kreuztabelle aus Python einfügen
% \input{tables/ep01_fe14_crosstab.tex}

\textbf{Interpretation:} [TODO: Beschreibung der gemeinsamen Verteilung]

\subsubsection{Grafische Darstellung}

\begin{figure}[h]
    \centering
    % TODO: Mosaikdiagramm aus Python einfügen
    % \includegraphics[width=0.85\textwidth]{../figures/block2/ep01_fe14_mosaic.pdf}
    \caption{Mosaikdiagramm: Zusammenhang zwischen aktueller Wirtschaftseinschätzung 
    (\texttt{ep01}) und Erziehungsziel „beliebt sein" (\texttt{fe14}).}
    \source{Eigene Darstellung auf Basis ALLBUS 2021.}
    \label{fig:ep01_fe14}
\end{figure}

\subsubsection{Zusammenhangsmaß}

Für den Zusammenhang zweier ordinaler Variablen eignet sich die Spearman-Rangkorrelation:

\begin{itemize}
    \item \textbf{Spearman-Korrelation:} $r_s =$ [TODO]
    \item \textbf{p-Wert:} [TODO]
\end{itemize}

\textbf{Interpretation:} [TODO: Stärke und Richtung des Zusammenhangs, Signifikanz]

\subsection{Aufgabe 2b: Korrelation \texttt{xt10} und \texttt{age} (metrisch)}

\subsubsection{Streudiagramm}

\begin{figure}[h]
    \centering
    % TODO: Streudiagramm aus Python einfügen
    % \includegraphics[width=0.85\textwidth]{../figures/block2/xt10_age_scatter.pdf}
    \caption{Streudiagramm: Zusammenhang zwischen Alter (\texttt{age}) und 
    Interviewdauer (\texttt{xt10}).}
    \source{Eigene Darstellung auf Basis ALLBUS 2021.}
    \label{fig:xt10_age}
\end{figure}

\subsubsection{Pearson-Korrelation}

% TODO: Korrelationstabelle aus Python einfügen
% \input{tables/xt10_age_correlation.tex}

\begin{itemize}
    \item \textbf{Pearson-Korrelation:} $r =$ [TODO]
    \item \textbf{p-Wert:} [TODO]
    \item \textbf{Stichprobengröße:} $n =$ [TODO]
\end{itemize}

\textbf{Interpretation:} [TODO: Stärke und Richtung des linearen Zusammenhangs, 
statistische Signifikanz]

\subsection{Aufgabe 2c: Korrelation ist nicht Kausalität}

Die gefundene Korrelation zwischen Alter und Interviewdauer [TODO: positiv/negativ/nicht signifikant] 
bedeutet \textbf{nicht}, dass das Alter die Interviewdauer verursacht (oder umgekehrt).

\subsubsection{Mögliche alternative Erklärungen}

\begin{enumerate}
    \item \textbf{Drittvariablen:} Eine nicht beobachtete Variable könnte beide beeinflussen. 
    Beispielsweise könnte Bildungsniveau sowohl mit Alter als auch mit der Ausführlichkeit 
    der Antworten korrelieren.
    
    \item \textbf{Interviewer-Effekte:} Die Interviewdauer hängt stark vom Interviewer ab 
    (Fragetempo, Gesprächsführung). Wenn bestimmte Interviewer bevorzugt ältere oder jüngere 
    Personen befragen, entsteht eine Scheinkorrelation.
    
    \item \textbf{Umgekehrte Kausalität:} Auch wenn eine kausale Beziehung bestünde, ist unklar, 
    in welche Richtung sie verläuft.
    
    \item \textbf{Zufälligkeit:} Bei einem Signifikanzniveau von 5\% ist jede 20. Korrelation 
    zufällig signifikant, selbst wenn kein echter Zusammenhang besteht.
\end{enumerate}

\subsubsection{Bedingungen für Kausalität}

Um eine kausale Aussage zu rechtfertigen, wären erforderlich:

\begin{itemize}
    \item \textbf{Zeitliche Reihenfolge:} Die Ursache muss der Wirkung vorausgehen
    \item \textbf{Ausschluss von Drittvariablen:} Kontrolle konfundierender Faktoren
    \item \textbf{Experimentelles Design:} Randomisierte Kontrollstudien (hier nicht möglich)
    \item \textbf{Theoretische Plausibilität:} Mechanismus muss erklärbar sein
\end{itemize}

\textbf{Fazit:} Beobachtete Korrelationen in Querschnittsdaten sind wichtige Hinweise auf 
Zusammenhänge, erlauben aber \textit{per se} keine kausalen Schlussfolgerungen. Weiterführende 
Analysen (z.\,B. Längsschnittstudien, Experimente) wären nötig, um Kausalität zu belegen.
