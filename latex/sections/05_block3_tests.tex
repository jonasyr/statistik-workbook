% ============================================================
%          BLOCK 3: INFERENZSTATISTIK
% ============================================================

\section{Inferenzstatistik mit \texttt{lm02}}
\label{sec:block3}

\subsection{Aufgabe 3a: Grundgesamtheit vs. Stichprobe}

\subsubsection{Grundgesamtheit (Population)}

Die \textbf{Grundgesamtheit} im Kontext des ALLBUS 2021 umfasst die deutschsprachige 
Wohnbevölkerung in Privathaushalten ab 18 Jahren in Deutschland. Dies entspricht etwa 
70 Millionen Personen.

\subsubsection{Stichprobe}

Die \textbf{Stichprobe} besteht aus den tatsächlich befragten [TODO: N] Personen, die nach 
einem Zufallsverfahren ausgewählt wurden. Diese Stichprobe soll repräsentativ für die 
Grundgesamtheit sein.

\subsubsection{Inferenzstatistischer Ansatz}

Die Inferenzstatistik erlaubt es, von den Stichprobendaten auf die Grundgesamtheit zu 
schließen:

\begin{itemize}
    \item \textbf{Punktschätzung:} Der Stichprobenmittelwert $\bar{x}$ schätzt den 
    Populationsmittelwert $\mu$
    \item \textbf{Intervallschätzung:} Konfidenzintervalle geben einen Bereich an, in dem 
    $\mu$ mit bestimmter Wahrscheinlichkeit liegt
    \item \textbf{Hypothesentests:} Prüfen von Vermutungen über die Population
\end{itemize}

\subsection{Aufgabe 3b: Konfidenzintervall für \texttt{lm02}}

Variable \texttt{lm02} erfasst die tägliche Fernsehdauer in Minuten.

\subsubsection{Deskriptive Statistik}

% TODO: Statistik-Tabelle aus Python einfügen
% \input{tables/lm02_descriptives.tex}

\begin{itemize}
    \item \textbf{Stichprobenmittelwert:} $\bar{x} =$ [TODO] Minuten
    \item \textbf{Standardabweichung:} $s =$ [TODO] Minuten
    \item \textbf{Stichprobengröße:} $n =$ [TODO]
\end{itemize}

\subsubsection{95\%-Konfidenzintervall}

Das Konfidenzintervall wurde mit der t-Verteilung berechnet (da $\sigma$ unbekannt):

\begin{equation}
    \text{KI}_{95\%} = \bar{x} \pm t_{n-1; 0{,}975} \cdot \frac{s}{\sqrt{n}}
\end{equation}

\textbf{Ergebnis:} [TODO: untere Grenze] $< \mu <$ [TODO: obere Grenze] Minuten

\subsubsection{Interpretation}

Das 95\%-Konfidenzintervall bedeutet: Wenn wir das Stichprobenverfahren unendlich oft 
wiederholen würden, lägen in 95\% der Fälle die so konstruierten Intervalle den wahren 
Populationsmittelwert $\mu$ enthalten.

\textbf{Wichtig:} Es bedeutet \textit{nicht}, dass der wahre Wert mit 95\% Wahrscheinlichkeit 
in diesem spezifischen Intervall liegt (der wahre Wert ist fix, nur unbekannt).

\subsection{Aufgabe 3c: t-Test für zwei unabhängige Gruppen}

\subsubsection{Fragestellung}

Unterscheidet sich die durchschnittliche Fernsehdauer (\texttt{lm02}) zwischen Männern 
und Frauen (\texttt{sex})?

\subsubsection{Hypothesen}

\begin{itemize}
    \item $H_0$: $\mu_{\text{männlich}} = \mu_{\text{weiblich}}$ 
    (kein Unterschied in der mittleren Fernsehdauer)
    \item $H_1$: $\mu_{\text{männlich}} \neq \mu_{\text{weiblich}}$ 
    (Unterschied besteht, zweiseitig)
\end{itemize}

Signifikanzniveau: $\alpha = 0{,}05$

\subsubsection{Deskriptive Statistik nach Gruppen}

% TODO: Gruppentabelle aus Python einfügen
% \input{tables/lm02_by_sex.tex}

\subsubsection{Testergebnis}

% TODO: t-Test-Ergebnis aus Python einfügen
% \input{tables/lm02_ttest.tex}

\begin{itemize}
    \item \textbf{t-Statistik:} $t =$ [TODO]
    \item \textbf{Freiheitsgrade:} $df =$ [TODO]
    \item \textbf{p-Wert:} $p =$ [TODO]
\end{itemize}

\subsubsection{Interpretation}

[TODO: Entscheidung über H₀]

\textbf{Bei $p < 0{,}05$:} Wir verwerfen $H_0$ und schließen, dass ein statistisch 
signifikanter Unterschied in der mittleren Fernsehdauer zwischen den Geschlechtern besteht.

\textbf{Bei $p \geq 0{,}05$:} Wir können $H_0$ nicht verwerfen. Die Daten liefern keine 
ausreichende Evidenz für einen Unterschied.

\textbf{Praktische Relevanz:} [TODO: Größe des Unterschieds bewerten]
