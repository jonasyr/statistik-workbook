% ============================================================
%                       EINLEITUNG
% ============================================================

\section{Einleitung}

Dieses Workbook dokumentiert die statistische Analyse des ALLBUS 2021 Teildatensatzes 
im Rahmen des Moduls „Deskriptive und Inferenzstatistik" an der IU Internationale Hochschule 
im Wintersemester 2025.

Der ALLBUS (Allgemeine Bevölkerungsumfrage der Sozialwissenschaften) ist eine repräsentative 
Querschnittserhebung, die seit 1980 Einstellungen, Verhaltensweisen und Sozialstruktur der 
Bevölkerung in Deutschland erhebt. Der vorliegende Teildatensatz aus dem Jahr 2021 umfasst 
29 ausgewählte Variablen.

\subsection{Zielsetzung}

Ziel dieser Arbeit ist die systematische Anwendung deskriptiver und inferenzstatistischer 
Methoden auf reale Umfragedaten. Im Einzelnen werden folgende Analysen durchgeführt:

\begin{enumerate}
    \item \textbf{Univariate Deskription:} Analyse einzelner Variablen nach Skalenniveau
    \item \textbf{Bivariate Deskription:} Untersuchung von Zusammenhängen zwischen zwei Variablen
    \item \textbf{Inferenzstatistik:} Konfidenzintervalle und Hypothesentests
    \item \textbf{Regression:} Modellierung linearer Zusammenhänge
    \item \textbf{ANOVA:} Vergleich von Mittelwerten über mehrere Gruppen
\end{enumerate}

\subsection{Aufbau der Arbeit}

Nach dieser Einleitung folgt in Kapitel~\ref{sec:daten} eine Beschreibung des Datensatzes 
und der verwendeten Variablen. Die Kapitel~\ref{sec:block1} bis \ref{sec:block5} dokumentieren 
die fünf Analyseblöcke gemäß Aufgabenstellung. Kapitel~\ref{sec:diskussion} diskutiert die 
Ergebnisse kritisch, bevor Kapitel~\ref{sec:fazit} die Arbeit zusammenfasst.

Alle statistischen Analysen wurden in Python durchgeführt (Version 3.12.6). Die verwendeten 
Pakete umfassen \texttt{pandas}, \texttt{numpy}, \texttt{matplotlib}, \texttt{scipy} und 
\texttt{statsmodels}.
